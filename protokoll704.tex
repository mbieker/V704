\documentclass[11pt,ngerman,a4paper]{article}
%Gummi|061|=)
\usepackage{amsmath}
\usepackage{a4wide}
\usepackage{url}
\usepackage{amsthm}
\usepackage{amsbsy}
\usepackage{amssymb}
\usepackage{inputenc}
\usepackage{rotating} 
\usepackage{graphicx}
\usepackage{paralist}
\usepackage{selinput}
\usepackage{siunitx}
\usepackage{booktabs}
\sisetup{}
\SelectInputMappings{%
adieresis={ä},
germandbls={ß},
}
\title{\textbf{Versuch V704: Absorption von $\gamma$- und $\beta$-Strahlung}}
\author{Martin Bieker\\
		Julian Surmann\\
		\\
		Durchgef\"{u}hrt am 15.04.2014\\
		TU Dortmund}
\date{}
\usepackage{graphicx}
\begin{document}
\renewcommand\tablename{Tabelle}
\renewcommand\figurename{Abbildung}
\maketitle
\thispagestyle{empty}
\newpage
\clearpage
\setcounter{page}{1}


\section{Einleitung}
In diesem Versuch wird die Wechselwirkung energiereicher Strahlung mit Materie untersucht. Es wird $\gamma$-Strahlung als Beispiel für Photonenstrahlung und $\beta$-Strahlung als Beispiel für Teilchenstrahlung betrachtet.
\section{Theorie}
Während es bei der $\gamma$-Strahlung gelingt, ein allgemeingültiges exponentielles Absorptionsgesetz aufzustellen, ist die Wechselwirkung von $\beta$-Strahlung mit Materie zu vielfältig für eine einfache Gesetzmäßigkeit. Im Gegensatz zur $\gamma$-Strahlung treten bei den schnellen Elektronen meisst eine Vielzahl von Wechselwirkungen auf, bis die gesamte kinetische Energie der Teilchen umgewandelt wurde. Ein allgemeingültiges Absorptionsgesetz für die $\gamma$-Strahlung zu formulieren ist für den Umfang dieses Versuches zu kompliziert. Stattdessen soll hier die sogenannte Reichweite mit einem einfachen Messverfahren untersucht werden.
\subsection{$\gamma$-Strahlung}
\subsubsection{Wirkungsquerschnitt und Absorptionsgesetz}
Für die Betrachtung der $\gamma$-Strahlung muss zunächst der sogenannte Wirkungsquerschnitt definiert werden. \newline
Der Wirkungsquerschnitt $\sigma$ ist ein Maß für die Häufigkeit der Wechselwirkung. Der Wirkungsquerschnitt lässt sich als fiktive Zielscheibe $\sigma$ beschreiben, die jedem Partikel des Absorbers zugeordnet sind. In einer Schicht $dx$, die sich an der Stelle $x$ befindet, finden dann
\begin{equation}
dN = -N(x)n\sigma dx
\label{1}
\end{equation}
Reaktionen statt. Die Formel sagt aus, dass sich die Zahl der Teilchen, die auch auf die Materieschicht hinter $dx$ auftreffen, um dN abnimmt.
\begin{equation}
\int_{N_0}^{N(D)} \frac{dN}{N} = \int_0^D n\sigma dx
\label{2}
\end{equation}
Formel (\ref{2}) zeigt die Integration aller Schichten, die die Anzahl der Teilchen, die nach der Absorption noch übrig sind. Das bekannte Absorptionsgesetz lautet dann
\begin{equation}
N(D) = N_0e^{-n\sigma D}.
\label{3}
\end{equation}
Es ist streng gültig, wenn jedes Teilchen höchstens einmal mit der Materie reagiert. Der Exponentialfaktor wird oft als $\mu$ abgekürzt und wird als Absorptionskoeffizient bezeichnet.
Die Halbwärtsdicke $D_{\frac{1}{2}}$ ist gegeben durch
\begin{equation}
D_{\frac{1}{2}} = \frac{ln 2}{\mu}
\label{4}
\end{equation}
Die Berechnung des Wirkungsquerschnittes mit
\begin{equation}
\sigma = \frac{\mu}{n} = \frac{\mu M}{zN_L\rho}
\label{5}
\end{equation}
ist nur eine grobe Annäherung an die Realität.
\subsubsection{Ursprung und Eigenschaften der $\gamma$-Strahlung}
Wechseln angeregte Atomkerne in einen energetisch niedrigeren Zustand, wird die freiwerdende Energie als $\gamma$-Quant abgegeben. Bei den $\gamma$-Quanten handelt es sich um keine klassischen Teilchen, analog zum Licht. Die Energieniveaus eines Kerns sind sehr genau definiert, daher stellt das das $\gamma$-Spektrum eines Atomkerns ein Linienspektrum mit extrem scharfen Linien dar.
\subsubsection{Wechselwirkung von $\gamma$-Strahlung mit Materie}
Bei der Wechselwirkung eines $\gamma$-Quants mit Materie können viele Wechselwirkungsprozesse auftreten. Die bei den üblichen Energien wichtigsten Prozesse sind der Photo-Effekt, der Compton-Effekt und die Paarbildung.
\paragraph{Der Photoeffekt}
Wenn ein $\gamma$-Quant in Wechselwirkung mit einem Elektron eines Atoms tritt, wird der $\gamma$-Quant vernichtet und das Elektron aus seiner Bindung an den Kern entfernt. Dabei hat das Elektron die kinetische Energie $E_B$. $E_B$ ist die Teilenergie des $\gamma$-Quants, die nicht für das Lösen des Elektrons umgewandelt wurde. Daher tritt der Photoeffekt nicht bei $\gamma$-Quanten mit einer Energie, die kleiner als die Auslöseenergie ist, auf. Bei Anwendung des Impuls-Satzes stellt man fest, dass das Atom einen Teil des Quantenimpulses aufnehmen muss. Daraus folgt, dass die Absorption des Quanten in der innersten Schale und bei schweren Atomen am häufigsten sind. Die durch den Photoeffekt entstehenden Lücken in den Schalen werden durch den Übergang von Elektronen höherer Schalen aufgefüllt, dabei wird ein Röntgenquant oder ein Auger-Elektron emittiert.
\paragraph{Der Compton-Effekt}
Beim Compton-Effekt wird der $\gamma$-Quant an einem freien Elektron gestreut. Dieser Effekt verursacht eine Energie- und Richtungsänderung des gestreuten $\gamma$-Quantens. Aus dem Energie- und dem Impulssatz folgt zwar, dass der Quant nie seine ganze Energie abgeben kann, trotzdem nimmt die Intensität des $\gamma$-Strahls ab, da der Quant gestreut wird.
\begin{equation}
\sigma_{com} = 2\pi r_e^2\left(\frac{1+\epsilon}{\epsilon^2} \left( \frac{2(1+\epsilon)}{1+2\epsilon}-\frac{1}{\epsilon}ln(1+2\epsilon) \right) + \frac{1}{2\epsilon}ln(1+2\epsilon) - \frac{1+3\epsilon}{(1+2\epsilon)^2} \right)
\label{6}
\end{equation}
Formel (\ref{6}) zeigt den Wirkungsquerschnitt $\sigma_{com}$ für die Comptonstreuung in Abhängigkeit von der Quantenenergie. Dabei ist $\epsilon = E_{\gamma} / m_0c^2$ das Verhältnis der Quantenenergie $E_{\gamma}$ zur Ruheenergie des Elektrons. Der Absorptionskoefizient  $\mu_{com}$ eines Stoffes ergibt sich aus den Formeln (\ref{5}) und (\ref{6}):
\begin{equation}
\mu_{com} = n\sigma_{com}(\epsilon) = \frac{zN_L\rho}{M}\sigma_{com}(\epsilon).
\label{7}
\end{equation}
\paragraph{Die Paarbildung}
Die sogenannte Paarbildung kann auftreten, wenn die $\gamma$-Energie größer ist als die doppelte Ruhemasse des Elektrons ist.
Der Impulssatz zeigt, dass zusätzlich ein Teil des Quantenimpulses von einem weiteren Stoßpartner übernommen werden muss. Meistens übernehmen die Atomkerne des Absorbermaterials diesen Restimpuls. $\sigma_p \sim z^2$ lässt sich mit Hilfe der Quantenmechanik zeigen.
\newline\newline
Die Gewichtung der drei hauptsächlich aufgetretenen Effekte ist abhängig von der Quantenenergie. Abbildung \ref{a1} zeigt die Extinktionskoeffizienten am Beispiel des Absorbers Germanium. So ist bei den niedrigen Energien der Fotoeffekt dominierend, schon bei mittleren Energien kann er hingegen vernachlässigt werden. Der Compton-Effekt ist bei geringen Energien zwar vorhanden, spielt aber erst bei mittleren Energien eine Rolle. Die Paarbildung setzt erst bei 1 MeV ein, ist aber zunächst vernachlässigbar. Bei 100 MeV findet fast nur noch die Paarbildung statt.
\subsection{$\beta$-Strahlung}
\subsubsection{Entstehung und Eigenschaften der $\beta$-Strahlung}
$\beta$-Strahlung besteht aus positiven oder negativen Elektronen mit hoher kinetischer Energie. Diese Elektronen werden von Atomkernen mit einer instabilen Anzahl von Protonen oder Neutronen emittiert. Zusätzlich zu dem Elektron wird ein Antineutrino $\overline{\nu}_e$ bzw. das Neutrino $\nu_e$ emittiert. Da die Energie, die bei der Kernumwandlung frei wird, statistisch auf das Elektron, das Neutrino und auf den Rückstoßkern verteilt wird, ist das Spektrum eines $\beta$-Strahlers kontinuierlich. Die bei der Kernwandlung freigewordene Energie entspricht dabei der Maximalenergie $E_{max}$ des Elektrons. Durch das emittierte (Anti-)Neutrino wird die Einhaltung aller Erhaltungssätze gewährleistet.
\subsubsection{Wechselwirkung von $\beta$-Strahlung mit Materie}
Aufgrund ihrer Ladung und ihrer im Vergleich zu anderen Teilchen geringen Masse ist ein $\beta$-Teilchen für sehr viele Wechselwirkungen verantwortlich. Es treten vor allem drei Prozesse auf, die im Folgenden näher erläutert werden.
\paragraph{Elastische Streuung am Atomkern}
Bei dieser sogenannten Rutherford-Streuung werden die $\beta$-Teilchen im Coulombfeld der Atomkerne zum Teil mit großem Winkel abgelenkt. Durch diese Auffächerung der Teilchen wird die Intensität verringert. Darüber werden die Bahnlängen durch die Absorberschicht erheblich größer als die Reichweite der Elektronen. Dieser Effekt ist in Abbildung \ref{a2} dargestellt. Da die elastische Streuung unter relativistischen Geschwindigkeiten stattfindet, müssen die Rutherfordschen Streuformeln mehrfach korrigiert werden. Darauf soll hier aber nicht weiter eingegangen werden.
\paragraph{Inelastische Streuung am Atomkern}
Bei der Inelastischen Streuung wird Energie in Form von Bremsstrahlung abgegeben, da die Elektronen im Coulombfeld eines Kernes beschleunigt werden. Die Wahrscheinlichkeit dieses Prozesses wird bestimmt durch den Wirkungsquerschnitt $\sigma{Br}$. Bezogen auf den Atomkern ergibt sich
\begin{equation}
\sigma_{Br} = \alpha r_e^2 z^2.
\label{8}
\end{equation}
Folglich spielt die Bremsstrahlung nur bei schweren Atomkernen eine Rolle. Die mittlere in Bremsstrahlung umgewandelte Energie eines $\beta$-Teilchens bei Durchgang durch die Materieschicht beträgt
\begin{equation}
E_{Br}[keV] \approx 7*10^{-7}zE_{\beta}^2
\label{9}
\end{equation}
Allerdings ist diese Formel nur für Energien bis ca. 2500 keV gültig. Insgesamt hat die Bremsstrahlung nur einen kleinen Anteil an den Energieverlusten der $\beta$-Teilchen.
\paragraph{Inelastische Streuung an Elektronen des Absorbermaterials}
Bei der Absorption von $\beta$-Strahlung müssen Energieverluste bis zur Energie Null erklärt werden. Die daher notwendige dritte Art der Wechselwirkung ist die Inelastische Streuung an Elektronen. Durch diese Streuung werden die Kerne ionisiert und angeregt. Durch die nur geringe Umwandlung von Energie kann ein Elektron viele dieser Wechselwirkungen durchlaufen. Die Wahrscheinlichkeit dieser inelastischen Streuung ist proportional zur Anzahl der Elektronen pro Volumeneinheit. Bei Aluminium und einer Elektronenenergie von 100 keV werden $\beta$-Teilchen schon bei einer Schichtdicke des Aluminiums von ca. 0.15 mm vollständig abgebremst.
\subsubsection{Die Absorptionskurve von $\beta$-Strahlung}
Aufgrund der Komplexität der Absorptionsvorgänge bei $\beta$-Strahlung ist die Aufstellung des Zusammenhangs zwischen der Intensität der Strahlung und der Dicke des Absorbers ein hier nicht lösbares Problem. Allerdings zeigt sich, dass für kleine Schichtdicken und natürliche Strahler mit guter Näherung das Apsorptionsgesetz in Formel (\ref{3}) gilt. Ist die Schichtdicke allerdings zu nahe an der maximalen Reichweiter der Elektronen, kann dieses Gesetz nicht angewendet werden. Abbildung \ref{a3} zeigt eine typische Absorptionskurve. Die mit "Untergrund" bezeichnete Strahlenintensität setzt sich aus der Hintergrundstrahlung und der emitierten Bremsstrahlung der Elektronen zusammen. Um an der Absorptionskurve die maximale Reichweite der Elektronen zu bestimmen, verlängert man die beiden geraden Stücke des Graphen und liest die Reichweite im Schnittpunkt ab. Aus dieser Reichweite $R_{max}$ kann man die kernphysikalisch interessante Größe $E_{max}$ ermitteln. Aufgrund der so komplizierten Wechselwirkung ist der Zusammenhang aber nur empirisch. In unserem Experiment gilt
\begin{equation}
E_{max} = 1.92\sqrt{R_{max}^2 + 0.22 R_{max}}\, [MeV].
\label{10}
\end{equation}
\section{Vorbereitung und Durchführung}
\subsection{Aufbau}
Der grundlegende Versuchsaufbau ist der Abbildung \ref{a4} zu entnehmen. Ein Geiger-Müller-Zählrohr ist verstärkt an einem Zählwerk angeschlossen, die Messzeit ist einstellbar. Zwischen der strahlenden Probe und dem Zählrohr können Absorptionskörper in verschiedenen Stärken befestigt werden. Die Versuchsumgebung ist mit einer Bleiwand abgeschirmt.
\subsection{Durchführung}
Bevor die Probe eingesetzt wird, wird zunächst die Hintergrundstrahlung gemessen. Um eine hohe Genauigkeit zu erreichen, wird der Nulleffekt 900 Sekunden lang gemessen. In der folgenden Messung mit der Probe wird die Schichtdicke des Absorbers Schritt für Schritt erhöht, bei der Messung der $\beta$-Strahlung so lange, bis die Zählrate in der statistischen Schwankung konstant bleibt.\newline
Um eine geringere Strahlenbelastung zu erreichen, misst jede Praktikumsgruppe nur eine Strahlenart. Hier wird die $\beta$-Strahlung gemessen. Die Messwerte zur $\gamma$-Strahlung werden von unserer Partnergruppe bereitgestellt.
\section{Auswertung}
\subsection{Bestimmung der Messungenauigkeiten der Zählraten}
\subsection{Bestimmung des Absorptionskoeffizienten $\mu$ von $\gamma$-Strahlung}
Die Absorptionkurven wurden jeweils für eine Schicht aus Kupfer und einer Schicht aus Blei aufgenommen. Die Tabellen \ref{tab_gamma1}  und \ref{tabn_gamma_2} enthalten die über einen Zeitraum von 
\[
\Delta t_{Cu} = \SI{300}{\second}
\]
für Kupfer und
\[
\Delta t_{Pb} = \SI{100}{\second}
\]
für das Blei-Target gemessenen Werte und die entsprechenden Zählraten
\[
A = \frac{N}{\Delta t} - A_0.
\]
In den Abbildungen \ref{abb_gamma1_lin} und \ref{abb_gamma2_lin} sind die Absorptionskurven für die jeweiligen Metalle linear aufgetragen. Zur Berechnung der Absoprtionskoeffizienten wurden diese Werte auch halblogarithmisch aufgetragen (Abbildungen \ref{abb_gamma1_log} und \ref{abb_gamma2_log}) und eine lineare Ausgleichsrechung 
\[
\ln{A} = \mu \cdot d + \ln{A(0)}
\]
durchgeführt. Die folgenden Werte wurden bestimmt:
\begin{table}[h]
\centering
\begin{tabular}{rcc}

\toprule
	Material & $\frac{\mu}{\si{\meter^{-1}}}$ & $\ln{A(0)}$\\
 \midrule
	Cu & 22 & 23\\
	Pb & 32 & 33\\
\bottomrule
\end{tabular}
\end{table}

\noindent
Hieraus lässt sich durch Anwendung der Exponentialfunktion die Zählrate des Strahlers ohne Abschirmung $A(0)$ berechnen:
\begin{itemize}
\item $A_{Cu}(0)$ = \SI{}{\becquerel}
\item $A_{Pb}(0)$ = \SI{}{\becquerel}.

\end{itemize}
\paragraph{Theoretische Berechnung von $\mu$}
Zur Berechnung des Wirkungsqueschnitts $\sigma_{com}$ muss zunächst das Verhältnis von Quantenenergie zur Ruheenergie des Elektrons berechnet werden. Der Vewendete Strahler hat eine Quatenenergie von $E_\gamma = \SI{633}{\kilo\eV}$[Quelle]. Somit ergibt sich für 
\[
\epsilon =  \frac{E_\gamma}{m_ec^2} = \num{1.305}.
\]
Aus Formel \ref{6} berechnet $\sigma_{com}$ zu
\[
\sigma_{com} = \SI{2.76e-29}{\meter\squared}
\]
Aus diesem Wert kann aus Formel (\ref{7}). Dazu werden noch die Molare Masse $M$ sowie die Dichte $\rho$ des verwendeten Targets. Die folgende Tabelle zeigt die errechneten Werte:
\begin{table}[h]
\centering
\begin{tabular}{lllll}
\toprule
	Material & $Z$ & $\frac{M}{\si{\gram\per\mole}}$ &$\frac{\rho}{\si{\kilo\gram\per\meter\squared}}$ &$\frac{\mu}{\si{\meter^{-1}}}$ \\
\midrule
	Cu & 29 &63.546  &8.83 & 0.068\\
	Pb &82 & 207.2& 11.342 &     \\
\bottomrule
\end{tabular}

\end{table} 
\subsection{Bestimmung der $R_{Max}$ und der maximalen Energie der $\beta$-Strahlung}
Zur Bestimmung der Absorptionskurve von $\beta$-Strahlung wurde über einen Zeitraum von
\[
\Delta t = \SI{60}{\second}
\]
gemessen. Die Ergebnisse, sowie die dauraus berechnete Zählrate
\[
A = \frac{N}{\Delta t}
\]
befinden sich in Tabelle \ref{tab_beta}. In Abbildung \ref{abb_beta} werden diese Daten halblograrithmisch aufgetragen. Es zeigt sich der zu erwartende Kurvenverlauf. 
\begin{figure}[htp]
\centering
\includegraphics[scale=0.85]{/home/martin/Dokumente/SS14/Praktikum/V704/beta_log.png}
\caption{}
\label{abb_beta}
\end{figure}
\newpage
Eine lineare Augleichsrechnung
\[
y=m x +b
\]
wird jeweils für die ersten X Messpunkte sowie für die restlichen Daten ausgeführt. Daraus ergeben sich folgende Werte:
\begin{table}[h]
\centering
\begin{tabular}{cSS}
	\toprule
	Gruppe & $m$ & $b$\\
	\midrule
	A & -6.88+-0.19e+3 & 6.95+-0.06 \\
	B & -1.1+-2.6e+02 & 1.0+-0.4\\
	\bottomrule
\end{tabular}

\end{table}
Die X-Koordinate des Schnittpuntes der beiden Ausgleichsgeraden entspricht der maximalen Eindringtiefe $R_{Max}$. Hiermit ergibt sich:
\begin{equation}
R_{Max} = \frac{b_2-b_1}{m_1-m_2} = \SI{0.87+-0.08}{\milli\meter}.
\end{equation}

Aus diesem Wert kann mit Hilfe der empirschen Formel (\ref{10}) bestimmt werden.
\[
E_{Max} = 1.92\sqrt{R_{Max}^2+0.22R_{Max}}[\si{\mega\eV}] = \SI{0.0266+-0.0012}{\mega\eV}
\]

\section{Diskussion}

\section{Abbildungsverzeichnis}
\begin{enumerate}[{[}1{]}]
\item Entnommen der Praktikumsanleitung "Fourier-Analyse und Synthese" der TU Dortmund. Download am 18.01.14 unter:\\
 \url{http://129.217.224.2/HOMEPAGE/PHYSIKER/BACHELOR/AP/SKRIPT/V351.pdf}
\end{enumerate}
\section{Anhang}
\begin{itemize}
\item Tabellen
\item Auszug aus dem Messheft
\end{itemize}
\newpage

\begin{table}
\centering
\begin{tabular}{ScS}
\toprule
{{$\frac{d}{\si{\milli\meter}}$} } &{ $N$} &{ { $\frac{A-A_0}{\si{\second^{-1}}}$ } }\\
\midrule
3 & 9003 & 29.83+-0.32\\
5 & 8144 & 26.96+-0.30\\
8 & 7219 & 23.88+-0.28\\
10 & 6751 & 22.32+-0.27\\
13 & 5779 & 19.08+-0.25\\
15 & 5209 & 17.18+-0.24\\
18 & 4649 & 15.31+-0.23\\
\bottomrule
\end{tabular}
\label{tab_gamma1}
\caption{Ergebnisse der Messung des $\gamma$-Strahlers an einem Cu-Target.}
\end{table}



\begin{table}
\centering
\begin{tabular}{SSS}
\toprule
{{$\frac{d}{\si{\milli\meter}}$} } &{ $N$} &{ { $\frac{A-A_0}{\si{\second^{-1}}}$ } }\\
\midrule
1 & 3097 & 10.14+-0.32\\
10 & 1221 & 3.89+-0.30\\
11 & 1129 & 3.58+-0.28\\
20 & 495 & 1.47+-0.27\\
21 & 430 & 1.25+-0.25\\
30 & 192 & 0.46+-0.24\\
31 & 188 & 0.44+-0.23\\
\bottomrule
\end{tabular}
\label{Ergebnisse der Messung des $\gamma$-Strahlers an einem Pb-Target.}
\caption{}
\end{table}



\begin{table}
\centering
\begin{tabular}{SSS}
\toprule
{$\frac{d}{\si{\micro\meter}}$} &{ $N$} &{ $\frac{A}{\si{\second^{-1}}}$ }\\
\midrule
102 & 2.942+-0.017e+04 & 490.4+-2.9\\
126 & 2.578+-0.016e+04 & 429.7+-2.7\\
153 & 2.163+-0.015e+04 & 360.5+-2.5\\
160 & 1.984+-0.014e+04 & 330.6+-2.3\\
199 & 1.642+-0.013e+04 & 273.6+-2.1\\
252 & 1.197+-0.011e+04 & 199.4+-1.8\\
301 & 8.83+-0.09e+03 & 147.2+-1.6\\
338 & 6.19+-0.08e+03 & 103.1+-1.3\\
399 & 3.34+-0.06e+03 & 55.6+-1.0\\
444 & 2.60+-0.05e+03 & 43.4+-0.9\\
481 & 2.42+-0.05e+03 & 40.4+-0.8\\
483 & 2.38+-0.05e+03 & 39.6+-0.8\\
491 & 2.50+-0.05e+03 & 41.6+-0.8\\
\midrule
1008 & 214.0+-14.6 & 3.57+-0.25\\
1010 & 123.0+-11.1 & 2.05+-0.19\\
1527 & 140.0+-11.8 & 2.33+-0.20\\
1529 & 91.0+-10.0 & 1.52+-0.16\\
2019 & 144.0+-12.0 & 2.40+-0.20\\
2027 & 232.0+-15.2 & 3.87+-0.26\\
2451 & 102.0+-10.1 & 1.70+-0.17\\
\bottomrule
\end{tabular}
\label{tab_beta}
\caption{Ergebnisse der Messung des $\beta$-Strahlers an einem Al-Target.}
\end{table}



\end{document}