\documentclass[11pt,ngerman,a4paper]{article}
%Gummi|061|=)
\usepackage{amsmath}
\usepackage{a4wide}
\usepackage{url}
\usepackage{amsthm}
\usepackage{amsbsy}
\usepackage{amssymb}
\usepackage{inputenc}
\usepackage{rotating} 
\usepackage{graphicx}
\usepackage{paralist}
\usepackage{selinput}
\SelectInputMappings{%
adieresis={ä},
germandbls={ß},
}
\title{\textbf{Versuch V704: Absorption von $\gamma$- und $\beta$-Strahlung}}
\author{Martin Bieker\\
		Julian Surmann\\
		\\
		Durchgef\"{u}hrt am 15.04.2014\\
		TU Dortmund}
\date{}
\usepackage{graphicx}
\begin{document}
\renewcommand\tablename{Tabelle}
\renewcommand\figurename{Abbildung}
\maketitle
\thispagestyle{empty}
\newpage
\clearpage
\setcounter{page}{1}


\section{Einleitung}
In diesem Versuch wird die Wechselwirkung energiereicher Strahlung mit Materie untersucht. Es wird $\gamma$-Strahlung als Beispiel für Photonenstrahlung und $\beta$-Strahlung als Beispiel für Teilchenstrahlung betrachtet.
\section{Theorie}
Während es bei der $\gamma$-Strahlung gelingt, ein allgemeingültiges exponentielles Absorptionsgesetz aufzustellen, ist die Wechselwirkung von $\beta$-Strahlung mit Materie zu vielfältig für eine einfache Gesetzmäßigkeit. Im Gegensatz zur $\gamma$-Strahlung treten bei den schnellen Elektronen meisst eine Vielzahl von Wechselwirkungen auf, bis die gesamte kinetische Energie der Teilchen umgewandelt wurde. Ein allgemeingültiges Absorptionsgesetz für die $\gamma$-Strahlung zu formulieren ist für den Umfang dieses Versuches zu kompliziert. Stattdessen soll hier die sogenannte Reichweite mit einem einfachen Messverfahren untersucht werden.
\subsection{$\gamma$-Strahlung}
\subsubsection{Wirkungsquerschnitt und Absorptionsgesetz}
Für die Betrachtung der $\gamma$-Strahlung muss zunächst der sogenannte Wirkungsquerschnitt definiert werden. \newline
Der Wirkungsquerschnitt $\sigma$ ist ein Maß für die Häufigkeit der Wechselwirkung. Der Wirkungsquerschnitt lässt sich als fiktive Zielscheibe $\sigma$ beschreiben, die jedem Partikel des Absorbers zugeordnet sind. In einer Schicht $dx$, die sich an der Stelle $x$ befindet, finden dann
\begin{equation}
dN = -N(x)n\sigma dx
\label{1}
\end{equation}
Reaktionen statt. Die Formel sagt aus, dass sich die Zahl der Teilchen, die auch auf die Materieschicht hinter $dx$ auftreffen, um dN abnimmt.
\begin{equation}
\int_{N_0}^{N(D)} \frac{dN}{N} = \int_0^D n\sigma dx
\label{2}
\end{equation}
Formel (\ref{2}) zeigt die Integration aller Schichten, die die Anzahl der Teilchen, die nach der Absorption noch übrig sind. Das bekannte Absorptionsgesetz lautet dann
\begin{equation}
N(D) = N_0e^{-n\sigma D}.
\label{3}
\end{equation}
Es ist streng gültig, wenn jedes Teilchen höchstens einmal mit der Materie reagiert. Der Exponentialfaktor wird oft als $\mu$ abgekürzt und wird als Absorptionskoeffizient bezeichnet.
Die Halbwärtsdicke $D_{\frac{1}{2}}$ ist gegeben durch
\begin{equation}
D_{\frac{1}{2}} = \frac{ln 2}{\mu}
\label{4}
\end{equation}
Die Berechnung des Wirkungsquerschnittes mit
\begin{equation}
\sigma = \frac{\mu}{n} = \frac{\mu M}{zN_L\rho}
\label{5}
\end{equation}
ist nur eine grobe Annäherung an die Realität.
\subsubsection{Ursprung und Eigenschaften der $\gamma$-Strahlung}
Wechseln angeregte Atomkerne in einen energetisch niedrigeren Zustand, wird die freiwerdende Energie als $\gamma$-Quant abgegeben. Bei den $\gamma$-Quanten handelt es sich um keine klassischen Teilchen, analog zum Licht. Die Energieniveaus eines Kerns sind sehr genau definiert, daher stellt das das $\gamma$-Spektrum eines Atomkerns ein Linienspektrum mit extrem scharfen Linien dar.
\subsubsection{Wechselwirkung von $\gamma$-Strahlung mit Materie}
Bei der Wechselwirkung eines $\gamma$-Quants mit Materie können viele Wechselwirkungsprozesse auftreten. Die bei den üblichen Energien wichtigsten Prozesse sind der Photo-Effekt, der Compton-Effekt und die Paarbildung.
\paragraph{Der Photoeffekt}
Wenn ein $\gamma$-Quant in Wechselwirkung mit einem Elektron eines Atoms tritt, wird der $\gamma$-Quant vernichtet und das Elektron aus seiner Bindung an den Kern entfernt. Dabei hat das Elektron die kinetische Energie $E_B$. $E_B$ ist die Teilenergie des $\gamma$-Quants, die nicht für das Lösen des Elektrons umgewandelt wurde. Daher tritt der Photoeffekt nicht bei $\gamma$-Quanten mit einer Energie, die kleiner als die Auslöseenergie ist, auf. Bei Anwendung des Impuls-Satzes stellt man fest, dass das Atom einen Teil des Quantenimpulses aufnehmen muss. Daraus folgt, dass die Absorption des Quanten in der innersten Schale und bei schweren Atomen am häufigsten sind. Die durch den Photoeffekt entstehenden Lücken in den Schalen werden durch den Übergang von Elektronen höherer Schalen aufgefüllt, dabei wird ein Röntgenquant oder ein Auger-Elektron emittiert.
\paragraph{Der Compton-Effekt}
Beim Compton-Effekt wird der $\gamma$-Quant an einem freien Elektron gestreut. Dieser Effekt verursacht eine Energie- und Richtungsänderung des gestreuten $\gamma$-Quantens. Aus dem Energie- und dem Impulssatz folgt zwar, dass der Quant nie seine ganze Energie abgeben kann, trotzdem nimmt die Intensität des $\gamma$-Strahls ab, da der Quant gestreut wird.
\begin{equation}
\sigma_{com} = 2\pi r_e^2\left(\frac{1+\epsilon}{\epsilon^2} \left( \frac{2(1+\epsilon)}{1+2\epsilon}-\frac{1}{\epsilon}ln(1+2\epsilon) \right) + \frac{1}{2\epsilon}ln(1+2\epsilon) - \frac{1+3\epsilon}{(1+2\epsilon)^2} \right)
\label{6}
\end{equation}
Formel (\ref{6}) zeigt den Wirkungsquerschnitt $\sigma_{com}$ für die Comptonstreuung in Abhängigkeit von der Quantenenergie. Dabei ist $\epsilon = E_{\gamma} / m_0c^2$ das Verhältnis der Quantenenergie $E_{\gamma}$ zur Ruheenergie des Elektrons. Der Absorptionskoefizient  $\mu_{com}$ eines Stoffes ergibt sich aus den Formeln (\ref{5}) und (\ref{6}):
\begin{equation}
\mu_{com} = n\sigma_{com}(\epsilon) = \frac{zN_L\rho}{M}\sigma_{com}(\epsilon).
\label{7}
\end{equation}
\paragraph{Die Paarbildung}
Die sogenannte Paarbildung kann auftreten, wenn die $\gamma$-Energie größer ist als die doppelte Ruhemasse des Elektrons ist.
Der Impulssatz zeigt, dass zusätzlich ein Teil des Quantenimpulses von einem weiteren Stoßpartner übernommen werden muss. Meistens übernehmen die Atomkerne des Absorbermaterials diesen Restimpuls. $\sigma_p \sim z^2$ lässt sich mit Hilfe der Quantenmechanik zeigen.
\newline\newline
Die Gewichtung der drei hauptsächlich aufgetretenen Effekte ist abhängig von der Quantenenergie. Abbildung \ref{a1} zeigt die Extinktionskoeffizienten am Beispiel des Absorbers Germanium. So ist bei den niedrigen Energien der Fotoeffekt dominierend, schon bei mittleren Energien kann er hingegen vernachlässigt werden. Der Compton-Effekt ist bei geringen Energien zwar vorhanden, spielt aber erst bei mittleren Energien eine Rolle. Die Paarbildung setzt erst bei 1 MeV ein, ist aber zunächst vernachlässigbar. Bei 100 MeV findet fast nur noch die Paarbildung statt.
\subsection{$\beta$-Strahlung}
\subsubsection{Entstehung und Eigenschaften der $\beta$-Strahlung}
$\beta$-Strahlung besteht aus positiven oder negativen Elektronen mit hoher kinetischer Energie. Diese Elektronen werden von Atomkernen mit einer instabilen Anzahl von Protonen oder Neutronen emittiert. Zusätzlich zu dem Elektron wird ein Antineutrino $\overline{\nu}_e$ bzw. das Neutrino $\nu_e$ emittiert. Da die Energie, die bei der Kernumwandlung frei wird, statistisch auf das Elektron, das Neutrino und auf den Rückstoßkern verteilt wird, ist das Spektrum eines $\beta$-Strahlers kontinuierlich. Die bei der Kernwandlung freigewordene Energie entspricht dabei der Maximalenergie $E_{max}$ des Elektrons. Durch das emittierte (Anti-)Neutrino wird die Einhaltung aller Erhaltungssätze gewährleistet.
\subsubsection{Wechselwirkung von $\beta$-Strahlung mit Materie}
Aufgrund ihrer Ladung und ihrer im Vergleich zu anderen Teilchen geringen Masse ist ein $\beta$-Teilchen für sehr viele Wechselwirkungen verantwortlich. Es treten vor allem drei Prozesse auf, die im Folgenden näher erläutert werden.
\paragraph{Elastische Streuung am Atomkern}
Bei dieser sogenannten Rutherford-Streuung werden die $\beta$-Teilchen im Coulombfeld der Atomkerne zum Teil mit großem Winkel abgelenkt. Durch diese Auffächerung der Teilchen wird die Intensität verringert. Darüber werden die Bahnlängen durch die Absorberschicht erheblich größer als die Reichweite der Elektronen. Dieser Effekt ist in Abbildung \ref{a2} dargestellt. Da die elastische Streuung unter relativistischen Geschwindigkeiten stattfindet, müssen die Rutherfordschen Streuformeln mehrfach korrigiert werden. Darauf soll hier aber nicht weiter eingegangen werden.
\paragraph{Inelastische Streuung am Atomkern}

\section{Vorbereitung und Durchf\"{u}hrung}
Hier steht die Vorbereitung und die Durchführung.

\section{Auswertung}

\section{Diskussion}

\section{Abbildungsverzeichnis}
\begin{enumerate}[{[}1{]}]
\item Entnommen der Praktikumsanleitung "Fourier-Analyse und Synthese" der TU Dortmund. Download am 18.01.14 unter:\\
 \url{http://129.217.224.2/HOMEPAGE/PHYSIKER/BACHELOR/AP/SKRIPT/V351.pdf}
\end{enumerate}
\section{Anhang}
\begin{itemize}
\item Tabellen
\item Auszug aus dem Messheft
\end{itemize}

\end{document}